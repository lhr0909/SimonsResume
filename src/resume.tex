% LaTeX file for resume
% This file uses the resume document class (res.cls)
% \documentclass[margin]{res}

\documentclass[line]{res} % default is 10 pt

% the margin option causes section titles to appear to the left of body text
% \textwidth=5.2in % increase textwidth to get smaller right margin
% \usepackage{helvetica} % uses helvetica postscript font (download helvetica.sty)
% \usepackage{newcent}   % uses new century schoolbook postscript font

\setlength{\textheight}{10.2in} % increase text height to fit resume on 1 page
\topmargin=-0.5in % start text higher on the page
\newsectionwidth{0pt}  % So the text is not indented under section headings

\begin{document}

% the \\[12pt] adds a blank line after name
\name{Simon (Haoran) Liang}

\address{
  2801 1st Ave, Unit 613 \\
  Seattle, WA 98121 \\
  (812) 575-0373 \\
  simon@divby0.io \\
  \underline{http://www.divby0.io}
}
\address{\textbf{Open to Relocation}}

\begin{resume}

  \section{Education}
  \rule{\textwidth}{1pt}
    \textbf{Rose-Hulman Institute of Technology}, Terre Haute, IN
    \hfill September 2009 - May 2013 \\
    Bachelor of Science in Computer Science and Computer Engineering \\
    GPA \textbf{3.75/4.00} in Major, \textbf{3.53/4.00} Overall

  \section{Technical Skills}
  \rule{\textwidth}{1pt}
    Strong Language: Java, JavaScript (mithril.js, AngularJS, React, node.js), Python \\
    Knowledgeable Language: Ruby (on Rails), Objective-C \\
    Technology: AWS (S3, Lambda, SNS, SQS), OpenCV

  \section{Experience}
  \rule{\textwidth}{1pt}
    \textbf{Software Development Engineer}
    \hfill August 2013 - Present \\
    \textit{Amazon Fashion, Amazon, Inc., Seattle, WA}
    \begin{itemize} \itemsep -2pt  % reduce space between items
      % \item Started on the tech team for Javari subsidiary websites,
      %       then the team got absorbed into Amazon Fashion Tech Team
      \item Worked with a large variety of technologies as a full-stack developer
      \item Develop and maintain Amazon Fashion Pages on Amazon Website and Native iPad Application
      \item Develop and prototype enhanced UX web applications for fashion-engaged customers at Amazon
    \end{itemize}


  \section{Projects}
  \rule{\textwidth}{1pt}
    \textbf{Lead Developer}, Bra Fitter, Amazon Fashion, Seattle, WA
    \hfill July 2015 - Present \\
    \textit{Desktop Web Application that helps female customers find fitting bras}
    \begin{itemize} \itemsep -2pt  % reduce space between items
      % \item Application collects information about customer's current bra size and how the current bra(s) fit,
      %       then it adjusts the bra size using a fitting algorithm, and measured bra band sizes collected from
      %       various brands, and at the end, it recommends distinct fitting bra size down to brand level
      % \item Architected the survey frontend using isomorphic templates via \texttt{Mustache}
      \item Built frontend using a slim Virtual-DOM-Based JavaScript Framework,
            \texttt{mithril.js}, for uni-directional data flow
      \item Designed an AJAX-streaming technique to decrease the Above the Fold latency from \texttt{1500ms} to
            \texttt{250ms}, and made the service calls highly parallelized
		\end{itemize}

    \textbf{Developer}, Outfit Explorer Prototype, Seattle, WA
    \hfill January 2015 - June 2015 \\
    \textit{Responsive Web Prototype that allows customers to mix-and-match outfits
            via a drag-and-drop interface}
    \begin{itemize} \itemsep -2pt  % reduce space between items
      \item Helped design and prototype a \texttt{Backbone} Application driving a WebGL-enabled canvas view via
            \texttt{Pixi.js} in order to maximize rendering performance on all platforms
      \item Created a CRUD-based inventory management tool for merchandisers to upload cropped garment images onto S3
            using \texttt{AngularJS} and \texttt{Django}
      \item Prototyped a simple web-based garment cropping tool using \texttt{Django}, \texttt{OpenCV} and
            \texttt{AngularJS}, which later on ported to use \texttt{Java Spring}, \texttt{OpenCV} and
            \texttt{mithril.js} for performance, utilizing the GrabCut Algorithm
    \end{itemize}

    \textbf{Creator}, Simon’s Relocalizer for StarCraft II
    \hfill August 2012 - March 2013 \\
    \textit{StarCraft II – Online Multiplayer Real-Time Strategy PC Game}
    \begin{itemize} \itemsep -2pt  % reduce space between items
      \item Software modifies game client settings so players don’t need multiple copies to play on different regions
      \item Wrote software in \texttt{C\#} for Windows, hosted on GitHub as an open-source project
      \item Achieved 9,000 total downloads worldwide within one week, 20,000 total downloads within two weeks
      \item Had a total of 35,000 downloads worldwide, and was used by several professional StarCraft II players
      \item URL: \underline{https://github.com/lhr0909/SC2Patch150Relocalizer}
    \end{itemize}

  \section{Activities}
  \rule{\textwidth}{1pt}
    \textbf{Curriculum Planner, Mentor and Judge}, CodeGirls, Guangzhou, China
    \hfill December 2015 \\
    \textit{Bootcamp that teaches high school/college female students web development}
    \begin{itemize} \itemsep -2pt  % reduce space between items
      \item Designed a 5-day course on basic web development by teaching \texttt{Rails},
      \texttt{Bootstrap} and \texttt{jQuery}
      \item 12 Students worked on their website ideas in groups of 3 with the help of 3 mentors
      \item Wrote frontend tutorials and recorded videos for the write-ups
      \item URL: \underline{https://github.com/Code-Girls/2015Winter/wiki}
    \end{itemize}

    % \textbf{Intern Mentor}, Amazon Fashion, Seattle, WA
    % \hfill June 2015 \\
    % \textit{12-week Summer Internship Program at Amazon}
    % \begin{itemize} \itemsep -2pt  % reduce space between items
    %   \item Worked with the intern to gather requirements and choose appropriate
    %         technology for the project
    %   \item Gave advice on backend Java service development and dev resources
    %         for Android development
    %   \item Guided intern to learn and use Java 8 Stream API
    %   \item Suggested software engineering resources to read about refactoring and clean code
    %   \item Worked out plans with the intern to take the Android prototype to production
    % \end{itemize}

\end{resume}
\end{document}
